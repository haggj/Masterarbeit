\documentclass[../main.tex]{subfiles}

\begin{document}

\chapter{Terms and Definitions}
\section{AEAD}
\label{sec:aead}

\section{Key escrow}
\label{sec:key-escrow}

\section{Certificate revocation problem}
\label{sec:crp}

\section{Surreptitious Forwarding}
\label{sec:surreptitious-forwarding}
Qulle here~\cite{Davis2001}

\section{End-to-end encryption} 
\label{sec:end-to-end}

The National Institute of Standards and Technology (NIST) defines end-to-end encryption (E2EE) as follows:
\begin{quote}
"Communications encryption in which data is encrypted when being passed through a network, but routing information remains visible."~\cite[88]{Nieles2017}
\end{quote}
Consider the scenario, where Alice wants to send end-to-end encrypted message to Bob.
Once they established cryptographic keys, Alice encrypts the message and sends it to Bob.
During transit the encrypted message passes an untrusted network.
In a classical understanding of E2EE, no third party or adversary within the untrusted network is able to access the message because it is encrypted \cite{Ermoshina2016}.
If this holds, Alice and Bob are said to be the endpoints of the encryption and their communication is end-to-end encrypted.

\citeauthor{Hale2022} pointed out, however, that this classical understanding of E2EE has changed in recent years~\cite{Hale2022}. 
With the rise of modern instant messaging services, E2EE should not only guarantee the confidentiality of the communication (e.g. that the communication can not be decrypted by the untrusted network).
Rather, an adversary should not have any possibilities to influence the communication in any way.
This implies that a modern E2EE based system should also guarantee authenticity and integrity of the exchanged data~\cite{Hale2022}. 
The difference between the classical and current understanding of E2EE can be described more precise with different adversary models they defend against.

\subsection{Adversary model}
This section details the security guarantees E2EE can provide by considering different adversaries.
The above NIST definition states, that E2EE protects data passing an untrusted network.
Within the network, the communicating entities do not have any control over their data.
This motivates the idea, that the attacker is interpreted as the whole untrusted network:
\begin{quote}
    "The attacker carries the message" \todo{find source}
\end{quote}
It is assumed, that the adversary has limited computational capacity. 
Specifically, it can no break correctly implemented cryptographic schemes without knowing the key.
In the following, active and passive attackers are further distinguished. 

\subsubsection{Passive adversaries}
In general, passives attacks aim to break confidentiality.
This includes eavesdropping and unauthorized reading of data~\cite{Eckert2018}. 
In literature this is often defined as a honest-but-curious attacker. \todo{source + definition}
The classical understanding of E2EE protects against passive adversaries. 
The attacker is part of the untrusted network and observes the passed data.
Its goal is it to get access the content of the exchanged messages.
To defend against this passive attacker, cryptography can be applied to make passive attacks ineffective and to protect confidentiality~\cite{Eckert2018}.


\subsubsection{Active adversaries}
Todays interpretation of E2EE system is motivated by a stronger attacker model.
An active attacker not only listens to the passed traffic.
Rather, this attacker tries to insert, replace, drop, modify or replay messages to disrupt the communication.
More sophisticated active attacks also include spoofing or denial of service attacks.
The motivation can be diverse: 
The attacker might want to access confidential content, introduce faked data or delete valid data passing the network.
Another goal might be the impersonation of a user participating in the system.~\cite{Eckert2018}

Possible defense strategies against this adversary are getting more complex.
The detection of inserted, replaced or modified data require some sort of validation, that the received data was actually created by the assumed communication partner (authenticity) and that it was not modified during transit (integrity).
This can be realized with authenticated encryption techniques~\cite{Mallory2022}.
To defend against dropped or replayed messages advanced mitigation strategies must be applied:
Replayed messages can be detected with freshness, e.g. timestamps~\cite[419]{Eckert2018}.
Dropped messages, however, require some sort of sequence counter, e.g. implemented in the TCP protocol ~\cite[115]{Eckert2018}


\todo{Add references}
\cite{Mallory2022} understands it as active attacker (sec 2.4)
\cite{Hale2022} also proposes authenticate encryption which implicitly assumes an active attacker
\todo{Add references}
\cite{Nabeel2017} states, that if an attacker has access to metadata this is always critical when defending against active attackers, because metadata might allow to break privacy goals
Thus, he also proposes passive attackers against E2EE can also be understand as passive attacker (not authentication required)


\subsection{Endness}

An intuitive understanding of E2EE implies that nobody - except the encryption endpoints - has access to the plaintext data.
However, in an enterprise context this might not always be practical.
In order to encrypt and decrypt data, the communication endpoints need to access the required cryptographic keys.
If an attacker could compromise theses keys, the communication is not E2EE anymore, because the attacker could simply decrypt observed traffic with these keys.
Thus, only the employee is allowed to know its cryptographic keys.
Especially, they can not be shared or stored within the enterprise environment.
However, if the user looses his keys and the company does not have any backups all communication of the employee is lost.
This can be mitigated by introducing a trusted server, which stores cryptographic keys of all employees.
One the one hand, this seems to be practical and might increase availability of data. 
On the other hand, such a trusted server introduces key escrow and the communication is not E2EE in a strict sense anymore.
The administrator of the trusted server has access to all keys.
Since this administrator is most likely also the administrator of the whole enterprise network, he has access to the network traffic.
This allows him to decrypt the communication between employees.
Thus, the introduction of the trusted server most likely breaks E2EE.

However, there is also the possibility to explicitly define the administrator of the trusted server as an additional encryption endpoint.
In this case, there is no violation against E2EE, because the administrator is assumed to be a valid communication partner.
This example shows the fragility of E2EE.
By defining arbitrary encryption endpoints almost each system could claim to rely on E2EE.
While this is technically true, it does not meet the intuitive understanding of end-to-end encrypted systems.
\todo{cite HaHale2022 about endness}
\todo{more sources?}

\subsection{E2EE in this work}
This section defines the understanding of E2EE in the context of this thesis based on the above considerations.
Following the argumentation of~\citeauthor{Hale2022}~\cite{Hale2022}, E2EE is understood as defense against an active adversary.
The attacker not only observes the data, but it also tries to modify existing data or to insert forged data.
Any defense against this attacker requires both confidentiality and mutual authenticity.
The following example justifies this statement.
Assume a sender that wants to send an E2EE message to a single receiver.
The following points need to be ensured.
\begin{enumerate}
    \item Confidentiality: 
    The exchanged data needs to be encrypted. Otherwise each untrusted entity can access the exchanged data and E2EE is broken.
    Moreover, the sender needs to be sure that only the intended receiver can decrypt the data. 
    This is important in the context of symmetric cryptography, where a group of users might share a symmetric key.
    Data which is encrypted under this shared symmetric key can be decrypted by all members of the group.
    Thus, the encryption endpoints is equal to the group of users.
    In symmetric cryptography, E2EE messages between a sender and a single receiver always require a pairwise exchanged symmetric key.
    When operating with asymmetric cryptography, a key-pair per receiver is required.
    If a receiver keeps the private key confidential, it can be ensured that only the receiver can decrypt data encrypted under his or her public key.
    \item Authenticity: 
    The receiver needs to be sure that the message was indeed sent by the assumed sender. 
    This requirement is crucial to defend against man-in-the middle attackers.
    If a receiver can not validate the origin of the message, an attacker could simply impersonate the sender and encrypt faked data under the public key of the receiver.
    This requirement demands cryptographically signed messages to ensure that the message was actually sent by the claimed sender.
    Depending on the implemented E2EE protocol the receiver might be required to verify that the sender intended to share the data with him or her.
    Due to the risk of surreptitious forwarding this requires additional security mechanisms besides a plain cryptographic signature.
\end{enumerate}




% In contrast to this rather abstract definition, \citeauthor{Hale2022}~\cite{Hale2022} propose a very precise notion of E2EE.
% Based on a common normative understanding of end-to-end security the authors provide a formalism supporting practical system design.
% In the following it is reasoned, why the NIST definition is too vague to meet intuitive expectations of end-to-end security.

% First of all, it is not explicitly clear what kind of encryption is required.
% Intuitively the encryption should consider not only confidentiality but also integrity and authenticity~\cite{Hale2022}.
% Consider the scenario where Alice wants to send a message to Bob over an insecure network.
% Alice decides to encrypt her message with a secure encryption algorithm to ensure the confidential of her message.
% Since the message neither protects integrity nor authenticity a potential attacker in the network could easily modify, replace or add data to the communication.
% This does not meet the intuitive understanding of E2EE, since an attacker can successfully manipulate the communication~\cite{Hale2022}.
% A second limitation of the NIST definition is the lack of precise boundaries. 
% The encryption is restricted to the insecure network.
% Once the communication leaves the network the security guarantees are no longer enforced.
% A intuitive understanding of E2EE, however, implicates very precise encryption endpoints~\cite{Hale2022}.
% Usually these encryption endpoints are applications running on an end-user device.


% \citeauthor{Hale2022}~\cite{Hale2022} motivate a definition, where a system implementing E2EE needs to fullfil two essential requirements.
% First, the system must rely on an AEAD encryption scheme. 
% This provides confidentiality and authenticity of the encrypted data and allows to detect if the encrypted data was modified by an adversary.
% Second, the system must be very explicit about the endpoints of the communication. 
% Upon sending a message the sender needs to commit a set of valid endpoints.
% If there is any endpoint accessing the decrypted data, which the sender did not explicitly commit to, the communication is not considered to be E2EE.

\end{document}
