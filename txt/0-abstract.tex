\documentclass[../main.tex]{subfiles}

\begin{document}

\chapter*{\myAbstractTitle}

The concept of \emph{Inverse Transparency} aims to protect data sovereignty while enabling the sensible usage of data.
The \emph{transparency toolchain} supports this approach by providing a technical framework.
It ensures that all data accesses are transparently logged.
This thesis contributes to the toolchain by designing and implementing an end-to-end encrypted protocol that allows users to share and revoke access to their encrypted logs.
A survey of potential encryption techniques turns out that the elaborated requirements are fulfilled best when relying on hybrid encryption.
The designed protocol was implemented as a cryptographic library.
It is available in Go, Python, and Typescript.
It assumes that a secure PKI is available and relies on the "JSON Web Encryption and Signing" standard.

The existing toolchain was adapted to maintain encrypted logs.
The performance evaluation shows that the encryption layer introduces an acceptable overhead in the toolchain.
The encryption of a single log for up to 170 users does not introduce a human-noticeable delay.
If 100 encrypted logs are fetched from the server, the processing time increases by $55ms$ to $298ms$.
This shows that the designed protocol can be used in practice to share encrypted logs within the \emph{transparency toolchain}.

\end{document}