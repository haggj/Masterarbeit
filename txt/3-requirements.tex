
\documentclass[../main.tex]{subfiles}

\begin{document}


\chapter{Requirements}
\label{chap:requirements}

This chapter identifies the requirements of the protocol which needs to be implemented in the context of this thesis.
Overall, there are three major goals which need to be achieved. 

\begin{enumerate}[label=\Roman*.]
    \item Data owners need to be able to share (and revoke) their log data with other users in the system.
	\item Log data needs to be confidential by means of end-to-end encryption.
    \item Data owners should not be allowed to forge log data at any time.
\end{enumerate}


Traditional requirements engineering methods usually differentiate between functional and non-functional requirements based on the views of different stakeholders~\todo{sources}.
Since this thesis operates within the security domain, the traditional requirements engineering process needs to be adapted: 
Precise assumptions about what to protect and against whom to protect are mandatory for a successful implementation and evaluation.
Thus, this chapter closely follows the terms and methodology proposed by~\citeauthor{Fabian2010}~\cite{Fabian2010}.

The authors fundamentally differentiate between functional, non-functional and security requirements, which are extracted from functional, non-functional and security goals.
Goals are rather abstract and vague formulations of what the system should achieve. 
They are refined into more detailed requirements for two reasons.
First, verifiable requirements allow us to compare different approaches and to finally choose one solution from the solution space. 
Second, they also help to verify and evaluate if the implemented solution satisfies the overall expectations.~\cite{Fabian2010}

While the overall goal I will be treated as a functional goal (subsection \ref{functional-requriements}), the goals II and III are understood as security goals (subsection \ref{security-requriements}).
Further non-functional goals will be identified in subsection \ref{non-functional-requriements}.
Within each subsection the identified goals will be refined into requirements.
This finally results in a list of verifiable and consistent system requirements for this thesis (subsection \ref{system-requriements}).


\section{System requirements}\label{system-requriements}
This section summarizes all identified requirements elaborated in this chapter.
They were derived from the major goals above.
For details and the rationales of each requirement please have a look the corresponding subsections below.

\subsubsection{Functional requirements}
\begin{itemize}
    \item [F1.] Data owners can always access logs concerning them.
    \item [F2.] Data owners can share their logs with other users in the system.
    \item [F3.] Data owners can revoke the access of other users to their logs.
\end{itemize}

\subsubsection{Security requirements}
\begin{itemize}
    \item [S1.] Only the data owner and those explicitly given access to the logs can access them.
    \item [S2.] The users receiving a shared log can verify that the data owner has intended to share the log with them.
    \item [S3.] The users receiving a shared log can verify that the log was created by a valid monitor.
\end{itemize}

\subsubsection{Non-functional requirements}
\begin{itemize}
    \item [N1.] Minimal number of trusted entities
    \item [N2.] Standardized cryptographic algorithms
    \item [N3.] Usability
    \item [N4.] Minimal resource utilization
\end{itemize}

\section{Functional requirements}\label{functional-requriements}
Functional requirements describe \enquote{what the system does}~\cite[11]{Mylopoulos1992}.
They describe the accessability of logs in the inverse transparency toolchain.
Having access to a log effectively means that a user is able to decrypt the encrypted log.
In the context of this thesis, the following functional requirements are identified:
\begin{itemize}
    \item Data owners can always access logs concerning them.
    \item Data owners can share their logs with other users in the system.
    \item Data owners can revoke the access of other users to their logs.
\end{itemize}

\section{Security requirements}\label{security-requriements}
As proposed by~\citeauthor{Fabian2010}~\cite{Fabian2010}, this section determines the security goals of this thesis and refines them into security requirements.
Based on the major goals specified above and the identified functional requirements the following security goals have been identified:
\begin{itemize}
    \item Only the data owner and those explicitly given access to the logs can access them.
    \item The users receiving a shared log can verify that the data owner has intended to share the log with them.
    \item The users receiving a shared log can verify that the log was created by a valid monitor.
\end{itemize}

These security goals can be refined into security requirements by annotating them with additional information, a counter-stakeholder and specific circumstances~\cite{Fabian2010}.
Following this approach, each security requirement is specified by providing theses details.
The counter-stakeholder is an adversary attacking the system.
Satisfying a security requirement should effectively defend against this adversary.
The specified circumstances elaborate additional conditions which affect the security goal.
The following elaboration is intended to refine the security goals into concrete security requirements.

\subsubsection{Only the data owner and those explicitly given access to the logs can access them.}
In the context of the inverse transparency toolchain, the intended end-to-end encryption should ensure that only authorized users can access logs.
As specified in the functional requirements, the data owner can always decrypt the log.
The data owner can also share the log with other users which are then able to decrypt.
Authorized users are the data owner of the log plus all users the data owner has given access.
From the perspective of E2EE, all other entities are untrusted.
Untrusted entities must not be able to decrypt logs.
The untrusted network includes the whole infrastructure of the toolchain, except the application a user interacts with.
This implies, that all front-end applications must trusted (e.g. the \emph{Clotilde UI} and all integrated \emph{Monitor} components).

The following example attack motivates this requirement.
Consider an attacker Eve who is the database admin of the Overseer server.
Eve wants to know which monitor accesses which data frequently.
Thus, he logs into the server and tries to extract this information from the database.
If the stored logs are not encrypted, Eve can learn anything about the logs he might be interested in.
This example demonstrates how a passive attacker could access the content of all logs stored in the toolchain.
The attacker is passive because Eve does not participate in the protocol.
He simply observes the stored data.
The implemented protocol should defend against this attacker.

\subsubsection{The users receiving a shared log can verify that the data owner has intended to share the log with them.}

The functional requirements require, that only the data owner of a log is allowed to share a log.
No other user should be able to share the log in the name of the data owner.
Whenever a user receives an encrypted log it needs to be sure that the originator of the cipher is the data owner of the encrypted log.
Moreover, the receiver also needs to be sure that the data owner actually has intended to share the log with him or her.

The following exemplary attack motivates this requirement.
Assume Alice to be a data owner and Eve to be a malicious user.
Further assume that Alice shares a access log with Eve.
Eve now has valid access to the log and can decrypt the encrypted log.
Eve could itself apply the encryption algorithm and could specify Bob as receiver.
If the exchanged message does not contain information about the sender and the intended recipients, there is no way for Bob to know who actually sent the message and if the log was intentionally shared with him.
Bob might assume that Alice sent the message because she is the data owner of the received log.
Moreover, Bob could implicitly assume that he is a valid receiver because he can decrypt the message.
In this scenario, however, Alice did not want to share the log with Bob.
The log was simply forwarded by Eve without permission.
This attack demonstrates how an active attacker could share logs with unintended users.
The attacker forwards a valid log to an arbitrary user in the system.
The implemented protocol in the context of this thesis must defend against this attack.

\subsubsection{The users receiving a shared log can verify that the log was created by a valid monitor.}

A data owner can share logs with others within the system.
However, this introduces the risk that a malicious data owner shares forged logs with other users by manipulating existing logs or creating arbitrary new logs.
The manipulation of logs can be detected by integrity protection mechanisms.
Unfortunately, this does not prevent a data owner from create new logs with a valid integrity checksum.
A stronger defense is necessary here because any user who receives a log needs to verify who the creator of this log is.
Only a log of a trusted and valid creator should be accepted.

The following example motivates this requirement.
Assume Eve to be a malicious data owner.
Thus, he has access to all logs concerning him.
Eve wants to harm a specific data consumer.
To do so, Eve modifies an existing log or he creates a new log.
This log indicates an illegal behavior of a targeted data consumer.
Eve applies the encryption algorithm and shares the forged log with other users.
Without further protection the receiving users can not verify the creator of the log.
There is no way to detect this fraud.
This example demonstrates how a malicious data owner could insert forged log into the system.
The implemented protocol must defend against this attack.

\section{Non-functional requirements}\label{non-functional-requriements}
Non-functional requirements can be characterized as \enquote{global requirements on its development or operational costs, performance, reliability, maintainability, portability, robustness and the like}~\cite[11]{Mylopoulos1992}.
In contrast to the functional and security requirements, the non-functional requirements are not a direct result of the overall requirements of the thesis.
Thus, they are not strictly mandatory.
At the same time, however, they add additional value to potential solutions.
Consider two solutions $A$ and $B$, which both satisfy all functional and security requirements.
The solution $A$ should be preferred over a solution $B$, if $A$ resolves more non-functional requirements than $B$.

During multiple iterations of elaborating potential solutions, the list of non-functional goals has grown to the following stable collection:

\subsubsection{Minimal number of trusted entities}
By definition, trusted entities are not part of the untrusted network. 
Thus, the proposed E2EE (detail in section \ref{security-requriements}) does not defend against them.
This implies that a compromised trusted server undermines the security of the whole toolchain.
To reduce the risk and attack surface, the toolchain should rely on a minimal number of trusted entities.

\subsubsection{Standardized cryptographic algorithms}
The current implemented front-end of the inverse transparency toolchain runs within the browser. 
The Web Cryptographic API~\cite{WebCryptoApi2017} defines an interface to perform cryptographic operations within browsers. 
It \enquote{exposes already existing and often heavily verified cryptographic functionality to Web application developers through a standardized interface}~\cite[959]{Halpin2014}.
This enables secure and performant cryptography within the browser~\cite{Halpin2014}.
However, it is limited to specific primitives. 
Advanced encryption schemes are not yet defined and available in the API yet.
Specifically, advanced encryption schemes which are currently under research are not integrated.
To ensure portability of the inverse transparency toolchain, the implementation of this thesis should thus only rely on primitives defined by the Web Cryptographic API.
This is specifically important because the Clotilde front-end is a web application.

\subsubsection{Usability}
Usually a trade-off between Security and Usability needs to be resolved during system design~\todo{quelle}.
In the context of this thesis, however, the complexity of security is a key-driver to improve usability. 
By integrating the error-prone cryptography tasks (e.g. encryption or integrity checks) into the application, we can increase the usability and thus the acceptance of the toolchain.
At the same time, the security is improved because these tasks are performed within the toolchain and can be heavily tested.
However, this comes with the cost of increased design and implementation effort.
This approach is motivated by the principle of least effort, which states that humans will spent the least amount of work to get a task done~\cite{Levenson2018}.
As a result, an optimal solution should integrate all cryptographic tasks into the toolchain.
It should not demand users to perform them themselves.

\subsubsection{Minimal resource utilization}
This is a performance criterion.
The implemented solution should avoid resource overhead in terms of storage and computation.
While this is directly related to the non-functional requirement N2, more aspects need to be considered here.
Duplicated storage of data should be avoided.
Specifically, an optimal solution does not store the same data encrypted with different keys in the database to prevent redundancy.

\end{document}
