\documentclass[../main.tex]{subfiles}

\begin{document}

\chapter{Survey of potential solutions}

\section{External encryption}
\section{Mutual encryption}
\section{Key server}
\section{Broadcast encryption}
\section{Hybrid encryption}

\section{Further investigations}


\todo{Consider ZOOM E2EE}
Zoom operates within a live environment, e.g. all participants are online. 
Encrypting data in Zoom relies on a DH key exchange during encryption. 
This implies, that each data packet is encrypted with its own key enabling forward secrecy. \cite{Isobe2021}

\todo{Consider E2EE in IM messaging Apps}
Our implementation should provide privacy but not anonymity, which is implemented in modern instant messaging applications~\cite{Akinbi2021}. 
IM applications usually decrypt each message with its own fresh key and provide PFS in some cases (e.g. Double Ratchet protocol)

\todo{E2EE content distribution/file sharing}
CloudSeal relies on Proxy Re-Encryption techniques.
CloudSeal does not implement a Web application, but relies on native (OS) to implement the crypto \cite{Xiong2012}.
The scheme proposed by \citeauthor{Hoerandner2020} also implements a Proxy Re-Encryption scheme. \cite{Hoerandner2020}
There are also bindings for WASM which implement Prox Re-Encryption (https://github.com/IronCoreLabs/recrypt-rs), thus one could implement Proxy Re-Encryption schemes also within web applications.
Problem: Proxy Re-Encryption requires a semi trusted server.
If the server has a re-encryption key, it can simply re-encrypt all traffic for this entity.
This fundamentally breaks E2EE.


\end{document}
