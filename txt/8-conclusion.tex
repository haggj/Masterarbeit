\documentclass[../main.tex]{subfiles}

\begin{document}

\chapter{Conclusion and Discussion}
\label{chap:conclusion}

This final chapter concludes the thesis.
It summarizes the major results of the thesis in \cref{sec:result}.
It also discusses the limitations and potential future work in \cref{sec:limitations}.

\section{Results}
\label{sec:result}

This thesis designs, implements, and evaluates an end-to-end encrypted protocol that allows to share logs in the \emph{transparency toolchain}~\cite{Zieglmeier2021}.
It identifies the requirements that must be fulfilled by such a protocol.
The survey of encryption technologies shows that hybrid encryption meets those requirements best.
Therefore, the designed protocol relies on hybrid encryption.
The protocol was created to resist three types of attackers: 
A curious server cannot access the logs because all logs are encrypted.
Surreptitious forwarding attacks are detected since the user sharing a log cryptographically signs the log along with the indented receivers.
A malicious data owner cannot forge logs because each log must be cryptographically signed by a trusted monitor component.

The protocol was implemented in the programming languages Go, Python and Typescript.
For each of those, a library was developed, which can be used to sign, encrypt and decrypt logs.
These libraries enable the intended functionality of a multi-party E2EE protocol.
Moreover, each library is tested, documented, and published to the respective package indexes.

The existing toolchain was adapted to enable the functionality of sharing encrypted logs.
This is realized by including the implemented protocol into the toolchain.
A proof-of-concept environment demonstrates how an established PKI may be used to instantiate the protocol.
This verifies that the protocol can be used in practice to share encrypted logs among users.
From the perspective of a user, the encrypted logs can be visualized as before. 
Access to those logs can additionally be shared and revoked.
The implemented UI fully abstracts from any underlying cryptographic details such as digital signatures, encryption, and decryption algorithms.

The integration of the designed protocol into the toolchain verifies that the functional requirements can be fulfilled.
The security evaluation shows that the protocol can be considered secure under two assumptions.
First, a secure PKI must be available.
Second, the JOSE standard must describe secure algorithms.
The performance evaluation shows that the protocol introduces overhead due to the added encryption layer.
The encryption algorithm does not introduce a human-noticeable delay if the set of receivers does not contain more than $170$ users.
The bottleneck of the encryption layer occurs when multiple logs are downloaded from the server because this requires the decryption of all loaded logs.
If the front-end restricts the page size of the fetched logs to $100$, the processing time in the updated toolchain has increased by $55ms$ to $298ms$.
This indicates that the introduced overhead is kept within reasonable limits and can be handled in practice.

In conclusion, the designed and implemented protocol maintains end-to-end encrypted logs which can be shared within the system.
This protects the confidentiality of the stored logging data.
It also supports data owners that aim to legacy prosecute illegal data accesses.

\section{Limitations and future work}
\label{sec:limitations}
This thesis is designed to defend against three types of attacks.
They were derived from potentially malicious users participating in the sharing process of encrypted logs.
This, however, is not an exhaustive list.
The performed security analysis does not guarantee the general security of the protocol.
In particular, a third-party analysis of the applied techniques could further verify their security.
This could also include sophisticated mathematical proofs of the protocol as introduced by~\cite{Katz2020}.   
The security evaluation applied in this thesis at least indicates that the protocol can resist the assumed attackers if the respective assumptions are true.

The protocol tries to encrypt the logs because they might contain sensitive information.
However, the encrypted logs also reveal metadata to ensure that the encrypted data can be associated with users by intermediate servers.
This metadata could be subject to metadata analysis attacks, which might yield interesting information for an attacker.
Future work could investigate if this is a real threat.
The approach of encrypting logs still improves the current situation because it keeps the content of logs confidential.

This thesis assumes a secure PKI meaning that nobody knows the private keys of the users.
This is a strong assumption.
If the PKI is established by a trusted entity knowing the private keys of the users, the strict interpretation of E2EE is broken.
This is because in such a case not only the authorized users can decrypt the logs.
The knowledge of the private keys allows the trusted entity to decrypt and sign arbitrary data.
The protocol might be improved in the future for scenarios where the assumption of a secure PKI does not hold.

\end{document}
