\documentclass[../main.tex]{subfiles}

\begin{document}

\chapter{Conclusion}

This thesis designs, implements, and evaluates an end-to-end encrypted protocol that allows sharing logs in the \emph{Inverse Transparency} toolchain~\cite{Zieglmeier2021}.
The survey of encryption technologies shows that hybrid encryption meets the identified protocol requirements best.
Therefore, the designed protocol relies on hybrid encryption which enables efficient multi-party E2EE.
It makes it possible for data owners to share and revoke access to their logs while the data stays encrypted.
The protocol was created to resist three types of attackers: 
A curious server (e.g. the \emph{Safekeeper} component) can not access the logs because all logs are encrypted.
Surreptitious forwarding attacks are detected since the user sharing a log cryptographically signs the log along with the indented receivers.
A malicious data owner can not forge logs because each log is cryptographically signed by a trusted monitor component.

The protocol was implemented in the programming languages Go, Python and Typescript.
For each of those, a library was developed which can be used to sign, encrypt and decrypt logs.
These libraries enable the intended functionality of a multi-party E2EE protocol.
Moreover, each library is tested, documented, and was published to the respective package indexes.

The existing toolchain was adopted to enable the functionality of sharing encrypted logs.
The implemented protocol was therefore included in the toolchain.
A proof-of-concept environment demonstrates how an established PKI may be used to instantiate the protocol.
This verifies that the protocol can be used in practice to share encrypted logs among users.
From the perspective of a user, access to a log can be shared and revoked.
The implemented UI fully abstracts from any underlying protocol details such as digital signatures, encryption, and decryption algorithms.
The evaluation of the protocol shows that it is functional and secure.
Moreover, the implemented algorithms are usable in practice because data can be encrypted for up to 170 users without any noticeable delay.



\end{document}
